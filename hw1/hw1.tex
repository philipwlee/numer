\documentclass{article}

\usepackage{listings}
\usepackage{graphicx,tikz}
%\usepackage{pgfplots}
%\usepgfplotslibrary{polar}
%\usepgflibrary{shapes.geometric}
%\usetikzlibrary{calc}
\usetikzlibrary{arrows}
\usepackage{amsmath,amsthm}
\usepackage{amsfonts}
\usepackage{amssymb}
\usepackage{bbm}
\usepackage[letterpaper, portrait, margin=1in]{geometry}
\usepackage{diagbox}
\usepackage{slashbox}
\usepackage{tcolorbox}
\usepackage{listings}
\usepackage{dsfont}
\usepackage[scr=boondoxo,scrscaled=1.05]{mathalfa}
\newcommand{\Gal}{{\textrm{Gal}}}
\newcommand{\diam}{{\textrm{diam}}}
\newcommand{\cc}{{\mathbb C}}
\newcommand{\ee}{{\mathbb E}}
\newcommand{\id}{{\mathds{1}}}
\newcommand{\pp}{{\mathbb P}}
\newcommand{\rr}{{\mathbb R}}
\newcommand{\qq}{{\mathbb Q}}
\newcommand{\nn}{\mathbb N}
\newcommand{\zz}{\mathbb Z}
\newcommand{\aaa}{{\mathcal A}}
\newcommand{\bbb}{{\mathcal B}}
\newcommand{\rrr}{{\mathcal R}}
\newcommand{\fff}{{\mathcal F}}
\newcommand{\ppp}{{\mathcal P}}
\newcommand{\llll}{{\mathcal L}}
\newcommand{\ttt}{{\mathcal T}}
\newcommand{\jjj}{{\mathcal J}}
\newcommand{\sss}{{\mathcal S}}
\newcommand{\eps}{\varepsilon}
\newcommand{\vv}{{\mathbf v}}
\newcommand{\ww}{{\mathbf w}}
\newcommand{\xx}{{\mathbf x}}
\newcommand{\ds}{\displaystyle}
\newcommand{\Om}{\Omega}
\newcommand{\e}{e^{\frac{2\pi i}{3}}}
\newcommand{\eF}{e^{\frac{4\pi i}{3}}}
\newcommand{\CT}{\sqrt[3]{2}}
\newcommand{\inter}{\textrm{int}}
\newcommand{\R}{\textrm{Re }}
\newcommand{\I}{\textrm{Im }}
\newcommand{\pd}[2]{\frac{\partial #1}{\partial #2}}
\newcommand{\pdm}[3]{\frac{\partial^{#3}#1}{\partial #2^{#3}}}
\newcommand{\conj}[1]{\overline{#1}}
\newcommand{\Log}{\text{Log }}
\newcommand{\Res}[2]{\text{Res }\left[ #1 , #2 \right]}
\newcommand{\fcint}{\int_{-\pi}^\pi}
\newcommand{\vs}{\vspace{0.3cm}}
\newcommand{\vsf}{\vspace{0.4cm}}

\title{FINM 32000: Homework 1}
\author{Philip Lee (UCID \#12129240)}
\begin{document}
\maketitle

\begin{enumerate}
%%%%%%%%%%%%%%%%%%%%%%%%%%%%%%%%%%%%%%%%%%%%%%%%%%%%%%
% PROBLEM 1
%%%%%%%%%%%%%%%%%%%%%%%%%%%%%%%%%%%%%%%%%%%%%%%%%%%%%%
\item

\begin{enumerate}
	\item How should the integer j be chosen?
	
	Using $N=5\mathrm{e}+4$, $\Delta t =\frac{T}{N}= 5\mathrm{e}-6$. \\
	$\Delta x = \sigma \sqrt{3 \Delta t} = 1.549\mathrm{e}-3$.\\
	Therefore, $j \approx \log(\frac{114}{100}) \Delta x -0.5=84.078 \implies 84$ \\
	\\
	Using barrier\_trinom\_pricer, the barrier put price is $\$5.30$.
	
	\item $(K-S_T)^+ = (K-S_T)^+ \mathbbm{1}_{S_t<H} + (K-S_T)^+ \mathbbm{1}_{S_t \geq H}$ \\
	Therefore, we can price the up-and-in put using the up-and-out put and a vanilla European put. \\
	$P-P_{\mathbbm{1}_{S_t<H}} = P_{\mathbbm{1}_{S_t \geq H}}=\$0.22$
	
	\item
	\begin{enumerate}
	\item The continuously-monitored barrier option will be priced lower than or equal to the discretely-monitored. This is because the discrete monitoring has a chance of missing a crossing of the H threshold and not triggering the barrier whereas the cross will trigger for the continuously monitored.
	\item 
	$$\begin{aligned}
	S_t &= 114 \\
	\alpha &= \frac{C_t^{BS}(S_t,K=136.8)}{P_t^{BS}(S_t,K=95)} = \frac{5}{6} \\
	S_0 &= 100 \\
	P_0^{\text{cont}} &= P_0^{BS}(S_0,K=95) - \alpha C_0^{BS}(S_0,K=136.8) \\
		&= 5.03
	\end{aligned}$$
	\end{enumerate}
	
\end{enumerate}
\item 

\begin{enumerate}
	\item
	$$\begin{aligned}
	S=100, rGrow=r=0, K&=100 \\
	IV(C^{BS}=11.25, T=0.5)&=0.4001 \\
	IV(C^{BS}=12.00, T=1.0)&=0.3019
	\end{aligned}$$
	
	\item
	$$\begin{aligned}
	IV_{T=0.75} &= \frac{IV_{T=0.50}+IV_{T=1.00}}{2} \\
	&= 0.3510 \\
	sigma&=0.3510, S=100, \\
	r_{grow}&=0, r=0, K=100, T=0.75 \\
	C^{BS}&=12.08
	\end{aligned}$$
	
	\item $V = -C(T=0.75)+C(T=1.00) = -12.08 + 12.00 = 0.08$. Get Paid \$0.08 initially, non-negative payoff in the future: Type II arb. At $t=0.75$, if the options are ITM, exercise $C(T=1.00)=S_0.75-K$ to pay off $-C(T=0.75)=K-S_0.75$, whereas if the options are OTM, either sell $C(T=1.00)$ or hold it for longer to speculate. If cash settled, exercise for money to balance the account, if physically settled, exercise for shares to transfer to the buyer of $C(T=0.75)$. 

\end{enumerate}
\end{enumerate}

\end{document}
